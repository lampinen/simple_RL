\documentclass{beamer}
\usepackage{pgfpages}
%\setbeameroption{show notes on second screen=left} %enable for notes
\usepackage{graphicx}
\usepackage{xcolor}
\usepackage{listings}
\usepackage{hyperref}
\lstset{language=python,frame=single}
\usepackage{verbatim}
\usepackage{subcaption}
\usepackage{amsmath}
\usepackage{relsize}
\usepackage{appendixnumberbeamer}
\usepackage{xparse}
\usepackage{multimedia}
\usepackage{tikz}
\usetikzlibrary{matrix,backgrounds}
\pgfdeclarelayer{myback}
\pgfsetlayers{myback,background,main}

\tikzset{mycolor/.style = {line width=1bp,color=#1}}%
\tikzset{myfillcolor/.style = {draw,fill=#1}}%

\NewDocumentCommand{\highlight}{O{blue!40} m m}{%
\draw[mycolor=#1,rounded corners] (#2.north west)rectangle (#3.south east);
}

\NewDocumentCommand{\fhighlight}{O{blue!40} m m}{%
\draw[myfillcolor=#1,rounded corners] (#2.north west)rectangle (#3.south east);
}

\usetheme[numbering=fraction, background=dark]{metropolis}
%%\AtBeginSection[]
%%{
%%  \begin{frame}
%%    \frametitle{Table of Contents}
%%    \tableofcontents[currentsection]
%%  \end{frame}
%%}

%%\let\olditem\item
%%\renewcommand{\item}{\vspace{0.5\baselineskip}\olditem}
\begin{document}

\title{Reinforcement Learning 1}
\author{Andrew Lampinen}
\date{Psych 209, Winter 2018}
\frame{\titlepage}


\section{Introduction}
\begin{frame}{Chess}
\begin{columns}
\column{0.5\textwidth}
\begin{itemize}
    \item<1-> You're playing chess against Magnus Carlsen. How can you learn?
    \item<2-> It's not supervised, nobody is telling you what the right move was.
    \item<3-> Really the only definite signal you get is the game outcome.
    \item<4-> ... but have you ever had the feeling of ``Oh, f***'' after your opponent makes a move you didn't see?
\end{itemize}
\column{0.5\textwidth}
    \begin{center}
    \includegraphics[width = \textwidth]{figures/chess.jpg}
    \end{center}
\end{columns}
\note{How do you think we do it? Well, sometimes you make a move, and then Magnus moves, and then you're like OH I DIDN'T SEE THAT MY POSITION IS MUCH WORSE THAN I THOUGHT.}
\end{frame}

\begin{frame}{TD Learning}
\begin{itemize}
    \item<1-> Temporal Difference (TD) learning is a formal framework for learning from surprises. 
    \item<2-> Basic idea: keep a prediction about how things will go, and when you're suprised, revise it. 
    \item<3-> If Magnus surprises you with a forced checkmate, you should definitely revise your understanding of the previous position.
    \item<4-> What exactly are we predicting? 
    \begin{itemize}
        \item<5-> Simple approach: the \emph{value} of every position.
        \item<6-> If we have to re-evaluate a position, we can update our earlier prediction(s).
    \end{itemize}
\end{itemize}
\end{frame}

\begin{frame}{TD Learning in the brain}
    \begin{center}
    \includegraphics[height = 0.9\textheight]{figures/dopamine_TD.png}
    \end{center}
    {\tiny Schultz et al., 1997}
\end{frame}

\end{document}
